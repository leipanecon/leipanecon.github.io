% LaTeX Curriculum Vitae Template
%
% Copyright (C) 2004-2009 Jason Blevins <jrblevin@sdf.lonestar.org>
% http://jblevins.org/projects/cv-template/
%
% You may use use this document as a template to create your own CV
% and you may redistribute the source code freely. No attribution is
% required in any resulting documents. I do ask that you please leave
% this notice and the above URL in the source code if you choose to
% redistribute this file.

\documentclass[letterpaper]{article}
\usepackage{CJKutf8}
\usepackage{hyperref}
\usepackage{geometry}
\usepackage[super]{nth}
% Comment the following lines to use the default Computer Modern font
% instead of the Palatino font provided by the mathpazo package.
% Remove the 'osf' bit if you don't like the old style figures.
%\usepackage[T1]{fontenc}
%\usepackage[sc,osf]{mathpazo}

\begin{CJK}{UTF8}{gbsn}
% Set your name here
\def\name{陈志远}

% Replace this with a link to your CV if you like, or set it empty
% (as in \def\footerlink{}) to remove the link in the footer:
% \def\footerlink{http://jblevins.org/projects/cv-template/}

% The following metadata will show up in the PDF properties
\hypersetup{
  colorlinks = true,
  urlcolor  = blue,
  pdfauthor = {\name},
  pdfkeywords = {economics, statistics, mathematics},
  pdftitle = {\name: Curriculum Vitae},
  pdfsubject = {Curriculum Vitae},
  pdfpagemode = UseNone
}

\geometry{
  body={6.5in, 8.5in},
  left=1.25in,
  right=1.25in,
  top=1.25in
}

% Customize page headers
\pagestyle{myheadings}
\markright{\name}
\thispagestyle{empty}

% Custom section fonts
\usepackage{sectsty}
\sectionfont{\rmfamily\mdseries\Large}
\subsectionfont{\rmfamily\mdseries\itshape\large}

% Other possible font commands include:
% \ttfamily for teletype,
% \sffamily for sans serif,
% \bfseries for bold,
% \scshape for small caps,
% \normalsize, \large, \Large, \LARGE sizes.

% Don't indent paragraphs.
\setlength\parindent{0em}

% Make lists without bullets
%\renewenvironment{itemize}{
 % \begin{list}{}{
 %   \setlength{\leftmargin}{1.5em}
 % }
%}{
 % \end{list}
%}
\begin{document}

% Place name at left
%{\huge \name}

% Alternatively, print name centered and bold:
\centerline{\huge \bf \name}

\vspace{1ex}
\begin{center}
    {\scriptsize 最近更新:11/13/2019}
\end{center}
\normalsize
\vspace{0.25in}

\begin{minipage}{0.5\linewidth}
  {Pennsylvania State University} \\
  Department of Economics \\
  303 Kern Building \\
  State College, PA 16803
\end{minipage}
\begin{minipage}{0.5\linewidth}
  \begin{tabular}{ll}
    电话: & +1 (814) 883-0365 \\
    国籍: &  中华人民共和国 \\
    电子邮件: & \href{mailto:chenzhiyuan1224@gmail.com}{\tt chenzhiyuan1224@gmail.com} \\
    个人网页: & \href{https://zhiyuanchen.weebly.com/}{\tt https://zhiyuanchen.weebly.com/} \\
  \end{tabular}
\end{minipage}


%\section*{\textsc{Personal}}

%\begin{itemize}
%\item Born on February 8, 1991
%\item Married.
%\end{itemize}


\section*{\textsc{教育背景}}

\begin{minipage}{0.5\linewidth}
  经济学博士,\textbf{宾州州立大学} \\
  经济学硕士,\textbf{中国人民大学}  \\
  经济学学士,\textbf{中国人民大学}
\end{minipage}
\begin{minipage}{0.5\linewidth}
  2015-2020(预计)\\
  2013-2017\\
  2009-2013
\end{minipage}




%\section*{Employment}

%\begin{itemize}
%\item Stanford University 1927--1931.
%\item Columbia University 1931--1946.
%\item University of North Carolina, 1946--1973.
%\end{itemize}
\section*{\textsc{推荐导师}}
\begin{minipage}{0.5\linewidth}
    \begin{tabular}{l}
        \textbf{Jonathan Eaton (co-advisor)}  \\
         Distinguished Professor of Economics \\
         Phone: +1 (814) 867-3297 \\
         Email: \href{mailto:jxe22@psu.edu}{\tt jxe22@psu.edu}\\
           \\
         \textbf{Jingting Fan}\\
         Assistant Professor of Economics  \\
         Phone:+1 (814) 865-4592 \\
         Email: \href{mailto:jxf524@psu.edu}{\tt jxf524@psu.edu}
    \end{tabular}
\end{minipage}
\begin{minipage}{0.5\linewidth}
    \begin{tabular}{l}
   \textbf{Mark Roberts (co-advisor)} \\
     Liberal Arts Professor of Economics\\
 Phone:+1 (814) 863-1535 \\
 Email: \href{mailto:mroberts@psu.edu}{\tt mroberts@psu.edu}  \\
\\
   \textbf{James Tybout}  \\
Professor of Economics     \\
Phone: +1 (814) 865-4259  \\
Email: \href{mailto:jxt32@psu.edu}{\tt jxt32@psu.edu}
    \end{tabular}
\end{minipage}

\section*{学术研究}

\subsection*{\bf{研究方向}}
产业组织;宏观发展;国际贸易



\subsection*{\bf期刊发表}

\subsubsection*{英文期刊}
``Types of Patents and Driving Forces behind the Patent Growth in China," with Jie Zhang, {\textit{Economic Modelling}, 2019(80), 294-302.

\vspace{0.5em}
``Import and Innovation: Evidence from Chinese Firms," with Jie Zhang and Wenping Zheng, {\textbf{\textit{European Economic Review}}}, 2017(94), 205-220.

\vspace{0.5em}
``The Bank–firm Relationship: Helping or Grabbing?" with Yong Li and Jie Zhang, {\textit{International Review of Economics \& Finance}}, 2016(42), 385-403.

\vspace{0.5em}
``Do the Types of Subsidies and Firms’ Heterogeneity Affect the Effectiveness of Public R\&D Subsidies? Evidence from China’s Innofund Programme," with Fu Xin, Jie Zhang, and Xiaorong Du, \textit{Asian Journal of Technology Innovation}, 2016(24), 317-337.

\subsubsection*{中文期刊(部分)}
“出口与生产率关系的新检验:中国经验,” 与张杰,张帆合作,通讯作者,{\bf世界经济},2016(06), 54-76.

\vspace{0.5em}
“中国企业创新补贴政策绩效评估:理论与证据,” 与张杰,新夫,杨连星合作,通讯作者,{\bf 经济研究}, 2015(10),1-25.

\vspace{0.5em}
“进口与企业生产率:来自中国的经验证据,” 与张杰,郑文平合作,通讯作者,{\bf 经济学(季刊)},2015(4),1029-1052.

\vspace{0.5em}
“中国出口国内附加值的测算与变化机制,” 与张杰,刘元春合作, {\bf经济研究},2013(10), 124-138.

\subsection*{\bf{工作论文}}
“Finance, R\&D Investment, and TFP Dynamics,” \textbf{\textit{博士论文}}
 \vspace{0.05in}
 \begin{center}
 \begin{minipage}{0.9\linewidth}
   \textbf{摘要:} This paper investigates the role of R\&D investment in shaping the relationship between financial constraints and aggregate total factor productivity (TFP). I study a dynamic model in which R\&D investment, which affects productivity evolution endogenously, is subject to financial constraints. I parameterize the model with production, innovation, and balance sheet data. The estimated model implies sizeable \textit{static} TFP losses caused by capital misallocation and \textit{dynamic} TFP losses from distorting R\&D investment. The accumulation of internal funds reduces the static TFP loss gradually. In contrast, because R\&D has a persistent effect on productivity, the dynamic TFP loss rises initially and declines later. Compared to a model with exogenous productivity, innovation investment makes firms less able to use self-financing to reduce TFP losses, and prolongs the transition. Endogenous productivity growth amplifies the gains in TFP and output from financial reform, and leads to a longer-lasting consequence from a credit crunch. Improving the pledgeability of intangible assets in China to be the US level reduces the static TFP loss only 0.4\%, but the dynamic TFP loss by 7.1\%.
 \end{minipage}
 \end{center}
 \vspace{0.05in}

“A Cost-benefit Analysis of R\&D and Patents: Firm-Level Evidence from China,” \textit{R\&R} at \textbf{\textit {European Economic Review}}
 \vspace{0.05in}
 \begin{center}
 \begin{minipage}{0.9\linewidth}
   \textbf{摘要:} Building on a standard dynamic model of endogenous productivity change, I develop a flexible empirical framework to analyze the components of the returns to R\&D and quantify the patent value. Applying it to a sample of Chinese high-tech manufacturing firms, I quantify the short-run
and long-run benefits of R\&D investment. I find that around 70\% of the benefits of R\&D investment comes from non-patenting innovation. On average an invention (a utility model) patent causes 0.76 (0.67) percent increase in the firm value. The start-up costs of R\&D is estimated to be around ten times as large as the maintenance costs. The counterfactual analysis shows that financing start-up innovation is more effective than financing the maintenance costs in stimulating the R\&D investment.
 \end{minipage}
 \end{center}
 \vspace{0.05in}
 “Is Cracking Down on Corruption Really Good for the Economy? Firm-Level Evidence from a Natural Experiment in China,” 与金鑫,徐旭合作, \textit{R\&R} at \textbf{\textit{Journal of Law, Economics, \& Organization}}
  \vspace{0.05in}
 \begin{center}
 \begin{minipage}{0.9\linewidth}
   \textbf{摘要:} This paper investigates the negative consequences of anti-corruption measures on economic outcomes. By exploiting an unexpected corruption crackdown in Northeast China in 2004, we find that the crackdown significantly lowers firm productivity and reduces firm entry. The negative impacts are mainly borne by private and foreign firms while the state-owned firms are intact. We further find that private firms with personal connections fare worse than the state-owned firms, while unconnected private firms do not behave differently after the crackdown. Our findings suggest that cracking down on corruption may weaken some connected private firms' political ties and remove the "grease of the wheels" that supports firm development in weak market institutions.
 \end{minipage}
 \end{center}
 \vspace{0.05in}
 “Made and Created in China: Super Processors and Two-way Heterogeneity,” 与Aksel Erbahar,资源合作
   \vspace{0.05in}
 \begin{center}
 \begin{minipage}{0.9\linewidth}
   \textbf{摘要:} In this paper, we show that there exists a special breed of firms that are active in both ordinary and processing exports. Contrary to the existing literature that describes processing firms as inferior, these mixed firms are superior to other firms in multiple dimensions, and hence we call them “super processors.” We build on Antras et al. (2017) and Bernard et al. (2019) to develop a model in which firms are heterogeneous in multiple stages of production. Firms endogenously choose to become suppliers or final good producers, with those that excel in both manufacturing ability and blueprint quality engage in both activities. We test the theory by exploiting the adoption of the pilot “paperless” processing trade supervision programme by regional customs authorities in China that lowered the cost of processing trade but left ordinary trade costs unchanged. We find that facilitating processing exports induces productive domestic downstream firms to establish their own trademarks. Our results highlight that processing trade not only leads goods to be “Made in China,” but also “Created in China.”
 \end{minipage}
 \end{center}


 \vspace{0.05in}
 “Production Function Estimation for Multi-Product Firms,” 与廖漠雨合作
     \vspace{0.05in}
 \begin{center}
 \begin{minipage}{0.9\linewidth}
   \textbf{摘要:} We study a stylized model of multi-product firm with firm-product level heterogeneity in Hicks-neutral production technology. We characterize the empirical content of the model and show that the scale and location of the production function are non-parametrically non-identified without observing the allocation of inputs and exogenous input price variations. We develop the model's empirical content to be an estimation strategy for any parametric family of production functions. This procedure, however, suffers from unclear identification and the problem of computing optimal input allocations. In the case of Cobb-Douglas production function, we show the identification of the production function by obtaining a closed-form solution for unobserved input allocations as a function of product-level output quantities or revenues. Monte Carlo evidence shows that our identification strategy performs well. We then apply our methodology to the sector of agricultural goods manufacturing and find multi-product firms' production technology of multi-product differs from single-product firms even for the same product. We also find that multi-product firms produce its core (peripheral) products at higher (lower) technical efficiency than single-product firms. Lastly, we document a negative productivity spillover across different products within the multi-product firm.
 \end{minipage}
 \end{center}
  \vspace{0.05in}



\subsection*{\bf{已开展论文}}
 Bank Competition, Heterogeneous Defaulting Risks, and Margins of Innovation, 与张蒙博,张杰,新夫合作


\vspace{0.5em}
Technological Imports and In-house R\&D,与张杰,郑文平合作

\section*{\textsc{学术活动}}

\subsection*{\bf 教学与科研经历}
\begin{minipage}{0.8\linewidth}
  \textbf{讲师}, 宏观经济学原理, 宾州州立大学 \\
  \textbf{讲师}, 中级计量经济学, 宾州州立大学 \\
  \textbf{助教}, 中级微观经济学, 宾州州立大学 \\
  \textbf{暑期研究员},墨西哥中央银行
\end{minipage}
\begin{minipage}{0.2\linewidth}
    Summer 2019 \\
    Summer 2018 \\
   2017-2019  \\
  2017/07- 2017/09
\end{minipage}

\subsection*{{\bf期刊匿名审稿人}}
      \textit{Abacus, Applied Economics Letters, Asian Journal of Technology Innovation, China Economic Review, Economic Modelling, International Economics and Economic Policy, Journal of Management Science and Engineering, Emerging Markets Finance and Trade, Journal of Industry, Competition and Trade}

\subsection*{{\bf会议与讲座}}
{\footnotesize * 合作者参加}
\vspace{1em}

\textbf{2019:} The \nth{14} Annual Economics Graduate Student Conference (WUSTL), *Midwest International Economic Development Conference

\vspace{0.5em}
\textbf{2018:} *The first CICE Annual Conference (Tsinghua University), AEA Annual Meeting (poster session, Philadelphia), *Workshop in Applied and Theoretical Economics (WATE), *Special Issue Conference on “The Challenges of Managing and Modelling Innovation and Growth in China” (Renmin University)

\vspace{0.5em}
\textbf{2017:} China Meeting of the Econometric Society (Wuhan), *Annual Meeting of the International Consortium for China Studies (Mannheim)

\vspace{0.5em}
\textbf{2014:} Annual Geneva-China workshop (Geneva)


\section*{\textsc{荣誉奖励}}
张培刚发展经济学优秀论文奖,2018

\vspace{0.5em}
国际贸易学2016年十佳中文论文, 《世界经济年鉴2017》, 2017

\vspace{0.5em}
18届安子介国际贸易论文三等奖, 2014

\vspace{0.5em}
国家奖学金, 中国人民大学, 2014

\vspace{0.5em}
国家奖学金, 中国人民大学, 2013

\vspace{0.5em}
国家奖学金, 中国人民大学, 2012

\section*{\textsc{技能}}
\textbf{软件编程}: Matlab, Stata, Julia, Python

\vspace{0.5em}
\textbf{语言}: 中文 (母语), 英文 (流利)




\end{document}
